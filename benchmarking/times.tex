\documentclass[10pt]{article}

% Packages
\usepackage[margin=1in]{geometry}
\usepackage{graphicx}
\usepackage{amssymb}
\usepackage{enumerate}
\usepackage{colortbl}
\usepackage{tabularx}
\usepackage{longtable}
\usepackage{array}
\usepackage{booktabs}
\usepackage{arydshln}
\usepackage{xcolor}

\newcommand\VRule[1][\arrayrulewidth]{\vrule width #1}

% === Begin Document === %
\begin{document}

\begin{center}
\large {\sffamily {\textbf{Batched Inversion Results on SIDH Signatures (Yoo et. al)}}}\\
\normalsize {Robert Gorrie -- McMaster University -- \today}\\
\end{center}
\hline
\pagenumbering{arabic}


% === Content === %
\section{Batched Partial-Inversion Procedure}

describe how the procedure works\\
describe where the procedure can be used in SIDH/signatures\\

\section{Performance}

\subsection{Numbers}

All results are measured in clock cycles, executed on a single-core, 1.70 GHz Intel Celeron CPU. All benchmarks are averages computed from 100 randomized sample runs.

\begin{center}
\begin{tabular}{@{}lll@{}}
	\toprule
	Procedure & Perf. Without Batching & Perf With Batching \\
	\midrule
	KeyGen & 68881331 & 68881331\\
	Signature Sign & 15744477032 & 15565738003\\
	Signature Verify & 11183112648 & 10800158871\\
	\bottomrule
\end{tabular}
\end{center}

In the following table, "Batched Inversion" signifies running the batched partial-inversion procedure on 248 $\mathbb{F}_{p^{2}}$ elements. The procedure uses the binary GCD $\mathbb{F}_{p}$ inversion function which, unlike regular $\mathbb{F}_{p^{2}}$ montgomery inversion, is not constant time.\\

\begin{center}
\begin{tabular}{@{}ll@{}}
	\toprule
	Procedure & Performance \\
	\midrule
	Batched Inversion & 1721718\\
	$\mathbb{F}_{p^{2}}$ Montgomery Inversion & 874178\\
	\bottomrule
\end{tabular}
\end{center}

\subsection{Analysis}

check notes in "averages" file\\



\end{document}
%\includegraphics[width=\linewidth]{IMG_20160124_143724.jpg}

